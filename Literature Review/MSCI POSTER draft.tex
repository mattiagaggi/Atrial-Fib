
\documentclass[twocolumn, a1paper, 12pt]{article}
 %\usepackage[scale=0.78,size=a1]{beamerposter}
%\setlength{\paperheight}{33.1in}
%\setlength{\paperwidth}{23.4in}
\usepackage{cite}
\usepackage{graphicx}
\usepackage{amsmath}
\usepackage{commath}
\usepackage{adjustbox}
%\usepackage[sc]{mathpazo}
\usepackage[margin=0.5in]{geometry}
%\usepackage{multicol}
\usepackage{mathtools}
\usepackage{abstract}
\usepackage[compact]{titlesec}
\DeclareMathOperator{\sech}{sech}
\newcommand{\matr}[1]{\mathbf{#1}}
%\usepackage{placeins}
%\usepackage{textcomp}
\graphicspath{ {H:/} }

\usepackage{smartdiagram}
\usepackage{metalogo}
\usepackage{dtklogos}



\begin{document}

\title {\textbf{ A Sphere Model for Atrial Fibrillation}}
\author{Tigany Zarrouk \& Mattia Gaggi \textbf{Supervisor}: Kim Christensen \\ Condensed Matter Theory Group---Imperial College London}

\date{}
\maketitle






\section{\textbf{\underline{Introduction}}}

Atrial Fibrillation (AF) is a type of cardiac arrhythmia and one of the major causes of stroke and heart failure. AF is also the most widespread cardiac condition, with over 30 millions of people worldwide suffering from it. In the UK, expenses related to this disease account for more than $1\%$ of the total budget of the National Health Service, which is more than $800$ million pounds.Atrial fibrillation usually manifests itself in patients affected through alterations of the normal cardiac rhythm lasting short periods of time (also called fibrillation episodes).\\

Each heart beat originates as an electric signal in the sinoatrial (SA) node that propagates first into the atria, then through the atrioventricular (AV) node, through Purkinje fibres and finally from the ventricular endocardium to the ventricular epicardium. The electric signal is conducted in the cardiac muscle cells thanks to the polarisation mechanism of the cell membrane. 





%picture of the heart here

When the electric impulse propagates a single muscle cell goes through three stages:

\begin{itemize}
  \item \textit{Excitable state}: the muscle cell is at rest with a negative built up potential.
  										
  \item \textit{Excited state}: the muscle cell is excited by its excited neighbours and the Voltage of the cell is at its peak. 
  \item \textit{Refractory state} :the muscle cell goes through a phase during which it can't be excited again - this period is called \textit{refractory period)}.
\end{itemize}  

Where the refractory period depends on the last time a cell was excited and is always fixed to a constant.\\



AF is correlated to the amount of \textbf{ fibrosis} in the cardiac tissue which generates a process called reentry. \\
During reentry the electric signal wavefront propagation breaks. It has been shown a clear link between fibrosis and AF
Even though many treatments have been developed, the mechanisms behind this condition are not completely understood and AF remains a major topic in medical research. \\
The current research revolves around three possible approaches:


\smartdiagram[bubble diagram]{Current research,
 Physiological \\ models, Image-Based \\ Models, \textbf{ CA Models}, antiarrhythmic \\ drugs) }
 
 Our research is based on expanding on the wrok done by Kim Christensen on CA (Cellular Automata) models \cite{Christensen}.




\section{\textbf{Christensen \emph{et al.'s} Model}}

\subsection{Description of the model}

In the publication "A Simple Model for Identifying Critical Regions in Atrial Fibrillation"~\cite{Christensen}, Prof. Christensen produced a simple 2D model for AF. The cardiac cells were arranged in a squared grid with horizontal connections, with a percentage of \textbf{ transversal connections} (to reproduce the effect of fibrosis) and a percentage of \textbf{dysfunctional cells}.
Dysfunctional cells are excited by neighbours with a fixed probability.\\

\textbf{\textit{Propagation Rules the model}}\\

\begin{itemize}
  \item \textit{Excitable state}: at time step t=0, each cell at rest is an element in the grid 

  										
  \item \textit{Excited state}: at t=1, the cell is excited by one of its \textit{ 4 neighbour cells}, and it can excite one the excitable neighbour cells at the next time-step.
  \item \textit{Refractory state} :from t=2 onwards, the cell goes through the refractory state for t=T time-steps before becoming excitable again/
\end{itemize}  





\subsection{Results}

The result show that AF naturally arises from the model itself.
\textbf{ It is important to note that this model is the only model ever created where arrhythmia naturally occurs from the structure of the heart} instead of being put in by hand in some way (as for instance adding by hand a new source of electric impulses. 
 
 % insert picture here of the model with wavefronf broken etc



 In addition the simulations produced by the model show a correlation between fibrillation and fibrosis .
There is a threshold value of vertical connections beyond which AF is produced spontaneously (without changing refractory period or conduction speed). This matches real data according to which AF is correlated with fibrosis. 
The model also shows that burning the area affected by AF effectively stops the arrhythmia, which has been recently discovered to be the case. 


%insert picture 






\begin{block}{Implementing Restitution }

Even though in Prof.Christensen model, the refractory period of each muscle cell was always fixed, this is not what happens in a real heart.
In a real heart the refractory period depends on the heart beat rate and on the last time each cell it was excited. The relationship between refractory period and heart beat rate is called restitution.
\\
In the first part of our project, we implemented restitution in Kim Christensen model in order

\end{block}



\begin{block}{Model on a Sphere}



In research by Fedotov, a cellular automaton model was constructed on the surface of a triangulated sphere to model spiral rotor waves on a closed, heterogeneous surface \cite{Fedotov}. It was shown that both a decrease in refractory period and conduction velocity from a "healthy" myocardium state, where no irregular excitation waves above 300bpm are observed, resulted in an area where spiral waves were formed. In addition, a multi-electrode system was created to study the localization of AF sources by simulated electrograms. 


\end{block}


%\section{\textbf{\underline{Conclusion}}}




%In this literature review, possible underlying mechanisms of atrial fibrillation (AF) are described along with the relevant the electrophysiology of heart cells, specifically myocytes and  sinoatrial node cells. The integral role of these cells with regards to action potential propagation to contract the heart periodically is described. Models to simulate the action potentials of these cells to simulate reentry are seen to be a computationally costly way of modelling large scale cardiac tissue. Studied models of electrical wavefront propagation (reentry) that could cause AF, such as the ring, leading circle and spiral wave models are described; spiral wave reentry has been observed to be an underlying mechanism. Cellular automata---where cells' states change based on the states of their nearest neighbours---are viable models that reduce the amount of computation by not considering differential equations governing the action potentials. Therefore, reentry can be modelled on a large scale more easily. Moore neighbourhoods used with a Markus model can achieve smooth spiral waveforms that are isotropic and therefore model spiral wave reentry in a cellular automata models effectively. New directions of research can be taken to more accurately model AF such as: the morphology and heterogeneities of the heart, modelling 3D scroll wave dynamics and using hybrid I/O automata. 

\section{\textbf{\underline{Bibliography}}}
\begin{thebibliography}{7}

%####################################################


\bibitem{Comtois}
P. Comtois, J. Kneller, S. Nattel,
\emph{Of circles and spirals: Bridging the gap between the leading circle and spiral wave concepts of cardiac reentry}
Europace. 2005 Sep;7 Suppl 2:10-20.
DOI: http://dx.doi.org/10.1016/j.eupc.2005.05.011



\bibitem{Alonso}
S. Alonso, M. Bar and B. Echebarria
\emph{Nonlinear physics of electrical wave
propagation in the heart: a review},
Rep. Prog. Phys. 79 (2016) 096601, 
doi:10.1088/0034-4885/79/9/096601

\bibitem{Christensen}
Kim Christensen, Kishan A. Manani, and Nicholas S. Peters
\emph{Simple Model for Identifying Critical Regions in Atrial Fibrillation}
Phys. Rev. Lett. 114, 028104 – Published 16 January 2015



\bibitem{Fedotov}
N. M. Fedotov,  A. I. Oferkin, and S. V. Zhary,
\emph{Modeling Sources of Atrial Fibrillation on a Triangulated Sphere}
Biomedical Engineering, Vol. 49, No. 2, July, 2015, pp. 112115. 



%imref https://o.quizlet.com/i/1UZTM_D5PFefI8E5JOk8Ig_m.jpg


\end{thebibliography}


\end{document}